\section{Coordinator}\label{coordinator}
In the early stages of the Tanlge, the network is susceptible to several attacks. Some of these attacks are discussed in Chapter \ref{attacks}. In summary, a user that controls a majority of the hashing power can double-spend coins. Unlike in Bitcoin where a miner competes with all other miners, in IOTA an attacker only competes with nodes that actively issue transactions! Thus, in times where not many transactions are issued, an attack becomes more feasible to execute.

In order to protect itself against such attacks, the IOTA foundation operates a special node called the Coordinator (Coo). This node has a checkpoint function. By issuing periodically zero-value transactions, the Coo creates milestones. Every transaction that is directly or indirectly confirmed by this milestone is considered as valid. The Coo is a central entity in the tangle and manifests a single point of failure. The Coo is not able to invalidate transactions from previous milestones. However, the node has several privileges compared to a regular node.

\begin{enumerate}
    \item The Coo can prioritize transactions.
    \item The Coo has the ability to censor transactions by continuously not approving certain transactions.
    \item If Coo is attacked and no longer works, the entire network halts.
\end{enumerate}

As such a central element is not desired in a decentralized system, the IOTA foundations investigates in possible solutions to this problem. These solution statements are explained in Section \ref{coo-less-tangle}.