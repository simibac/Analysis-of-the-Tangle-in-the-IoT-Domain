\chapter{Summary and Conclusions}

The fact that every node in the network that wants to participate has to confirm other transactions, make the network more decentralized than most current blockchain implementations. The chosen signature scheme for quantum resistance makes sense with the emerging technological advancements in this area. 

However, the fact that the IOTA protocol uses PoW as an anti-spam mechanism, makes the network not practical for IoT devices. There are no solutions for the storage problem that do not affect the security of the network. Using the tangle for decentralized storage has problematic incentive structures as it is almost free to store but costs to read data. Furthermore, the tangle is prone to a single-point of failure as the coordinator is operated by a single party.

Due to these reasons, the current implementation of the protocol has too many flaws to build a real-world application which has to rely on the tangle.
